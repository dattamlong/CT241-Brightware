\phantomsection
\setsection{Chương 1: Tổng quan}
\setcounter{section}{1}

% \section*{\centering CHƯƠNG 1: TỔNG QUAN}
\phantomsection
\subsection{Mô tả về bài toán}
% Bài toán người giao hàng (Travelling Salesman Problem - TSP) là một trong những vấn đề tối ưu hóa kinh điển trong khoa học máy tính và toán học. Mục tiêu của bài toán này là tìm ra hành trình ngắn nhất mà một người giao hàng cần đi qua để ghé thăm tất cả các địa điểm (các điểm giao hàng) và trở về điểm xuất phát ban đầu, với điều kiện mỗi địa điểm chỉ được ghé thăm đúng một lần.

Bài toán cây khung nhỏ nhất (Minimum Spanning Tree - MST) là một bài toán cơ bản trong lý thuyết đồ thị, có ứng dụng rộng rãi trong các lĩnh vực như mạng máy tính, các hệ thống kết nối, viễn thông, và thiết kế mạch điện. Bài toán yêu cầu tìm cây khung nhỏ nhất của một đồ thị, tức là tìm ra một tập hợp các cạnh kết nối tất cả các đỉnh với nhau mà không hình thành chu trình, đồng thời có tổng trọng số của các cạnh là nhỏ nhất.

\phantomsection
\subsection{Mục tiêu của đề tài}
Đề tài này nhằm mục tiêu xây dựng một công cụ để giải quyết các bài toán tối ưu hóa như bài toán cây khung nhỏ nhất một cách nhanh chóng và chính xác. Đồng thời, đề tài sẽ xây dựng các giải thuật tối ưu dựa trên kiến thức đã học, mô tả chi tiết phương pháp giải quyết các bài toán theo thuật toán đã triển khai, và cung cấp giao diện đồ thị trực quan cho người dùng.

\phantomsection
\subsection{Hướng giải quyết và kế hoạch thực hiện}
Để giải bài toán cây khung nhỏ nhất, đề tài sẽ được thực hiện theo các bước chính sau:
\begin{itemize}
    \item \textbf{Phân tích bài toán:} Bài toán cây khung nhỏ nhất (MST) yêu cầu tìm một cây khung nối tất cả các đỉnh của đồ thị với tổng trọng số nhỏ nhất mà không tạo ra chu trình. Bài toán này có thể giải được trong thời gian đa thức và có ứng dụng trong việc tối ưu hóa mạng lưới giao thông, hạ tầng mạng, và giảm chi phí vận chuyển.
    \item \textbf{Nghiên cứu các thuật toán tối ưu:} Tìm hiểu và phân tích các thuật toán như Kruskal và Prim. Thuật toán Kruskal sắp xếp các cạnh theo trọng số và lần lượt chọn các cạnh nhỏ nhất mà không tạo chu trình, trong khi Prim bắt đầu từ một đỉnh và dần dần thêm các cạnh nhỏ nhất để kết nối các đỉnh chưa được thêm vào cây khung. Kruskal hiệu quả hơn với đồ thị thưa, còn Prim thường tốt hơn với đồ thị dày.
    \item \textbf{Cài đặt và triển khai các thuật toán:} Dựa trên các thuật toán Kruskal và Prim đã phân tích ở trên và tiến hành cài đặt, triển khai các thuật toán này. Các bước bao gồm định nghĩa đồ thị, triển khai thuật toán và tối ưu hóa cho các bài toán lớn.
    \item \textbf{Thử nghiệm và đánh giá hiệu suất:} Thực hiện thử nghiệm trên các đồ thị thưa và dày để đánh giá thời gian chạy của từng thuật toán. Sử dụng biểu đồ để so sánh thời gian thực thi và khả năng tối ưu của Kruskal và Prim.
    \item \textbf{So sánh các thuật toán:} So sánh hiệu suất của Kruskal và Prim trên các loại đồ thị khác nhau. Kruskal thường tốt hơn với đồ thị thưa, còn Prim lại vượt trội trên các đồ thị dày và khi cần xây dựng cây từ một đỉnh cụ thể.
    \item \textbf{Tổng kết và đề xuất cải tiến:} Đưa ra kết luận rằng cả hai thuật toán đều có thể sử dụng tùy theo loại đồ thị, nhưng có thể cải tiến bằng cách tối ưu hóa hàng đợi ưu tiên trong Prim hoặc sắp xếp cạnh trong Kruskal để giảm thời gian tính toán.
\end{itemize}

\textbf{Kế hoạch thực hiện đề tài sẽ được chia thành các giai đoạn sau:}
\begin{itemize} 
    \item Giai đoạn 1: Tìm hiểu và phân tích bài toán (tuần 4 - 5) 
    
    \item Giai đoạn 2: Nghiên cứu các thuật toán tối ưu, thiết kế giao diện người dùng (tuần 6 - 7) 
    
    \item Giai đoạn 3: Triển khai thuật toán (tuần 8 - 10) 
    
    \item Giai đoạn 4: Kiểm thử và hoàn thiện sản phẩm (tuần 11 - 14) 
    
    \item Giai đoạn 5: Tổng kết và viết quyển báo cáo (tuần 15) 
\end{itemize}