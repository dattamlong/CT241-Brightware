\fontsize{12pt}{14pt}\selectfont
\subsubsection*{\centering ĐÁNH GIÁ KẾT QUẢ THỰC HIỆN NIÊN LUẬN CƠ SỞ NGÀNH KTPM}
\begin{center}
  (Học kỳ: 01, Năm học: 2024 - 2025) \\
\end{center}
\vspace{-0.5cm}
\textbf{TÊN ĐỀ TÀI:} Ứng dụng các thuật toán tối ưu để phân công công việc cho nhân viên \\
\textbf{GIÁO VIÊN HƯỚNG DẪN} \\
\vspace{-0.9cm}
\begin{center}
  \begin{tabularx}{\textwidth}{|>{\centering\arraybackslash}X|>{\centering\arraybackslash}X|>{\centering\arraybackslash}X|}
    \hline
    STT & Họ và Tên              & MSCB \\
    \hline
    1   & Trương Thị Thanh Tuyền & 1068 \\
    \hline
  \end{tabularx}
\end{center}
\vspace{-0.5cm}
\textbf{SINH VIÊN THỰC HIỆN} \\
\vspace{-0.9cm}
\begin{center}
  \begin{tabularx}{\textwidth}{|>{\centering\arraybackslash}X|>{\centering\arraybackslash}X|>{\centering\arraybackslash}X|>{\centering\arraybackslash}X|}
    \hline
    \multirow{2}{*}{Họ và Tên} & \multirow{2}{*}{MSSV} & Thưởng                   & Điểm                     \\
                               &                       & \textit{(Tối đa 1 điểm)} & \textit{(Thang điểm 10)} \\
    \hline
    Nguyễn Tuấn Đạt            & B2203499              &                          &                          \\
    \hline
  \end{tabularx}
\end{center}
\vspace{-0.5cm}
\textbf{I. HÌNH THỨC} \textit{(0,5 điểm)} \\
\textbf{Bìa} \textit{(tối đa 0,25 điểm)}
\begin{itemize}[itemsep=-6pt, topsep=-6pt]
  \item Đầy đủ các thông tin
  \item Đúng định dạng
\end{itemize}
\textbf{Bố cục} \textit{(tối đa 0,25 điểm)}
\begin{itemize}[itemsep=-6pt, topsep=-6pt]
  \item Trang đánh giá kết quả thực hiện niên luận 1
  \item Mục lục: cấu trúc chương, mục và tiểu mục
  \item Phụ lục (nếu có)
  \item Tài liệu tham khảo
\end{itemize}
\textbf{II. NỘI DUNG} \textit{(3,5 điểm)} \\
\textbf{Giới thiệu} \textit{(tối đa 0,5 điểm)}
\begin{itemize}[itemsep=-6pt, topsep=-6pt]
  \item Mô tả bài toán \textit{(0,25 điểm)}
  \item Mục tiêu cần đạt, hướng giải quyết \textit{(0,25 điểm)}
\end{itemize}
\textbf{Lý thuyết} \textit{(tối đa 0,5 điểm)}
\begin{itemize}[itemsep=-6pt, topsep=-6pt]
  \item Các khái niệm sử dụng trong chương trình \textit{(0,25 điểm)}
  \item Kết quả vận dụng lý thuyết trong đề tài \textit{(0,25 điểm)}
\end{itemize}
\textbf{Ứng dụng} (tối đa 2 điểm)
\begin{itemize}[itemsep=-6pt, topsep=-6pt]
  \item Phân tích yêu cầu, xây dựng các cấu trúc dữ liệu \textit{(0.5 điểm)}
  \item Sơ đồ chức năng, lưu đồ giải thuật giải quyết vấn đề \textit{(1.0 điểm)}
  \item Giới thiệu sử dụng chương trình \textit{(0,5 điểm)}
\end{itemize}
\textbf{Kết luận} \textit{(tối đa 0,5 điểm)}
\begin{itemize}[itemsep=-6pt, topsep=-6pt]
  \item Nhận xét kết quả đạt được
  \item Hạn chế
  \item Hướng phát triển
\end{itemize}
\textbf{III. CHƯƠNG TRÌNH DEMO} \textit{(5 điểm)}\\
\textbf{Giao diện thân thiện với người dùng} \textit{(1,0 điểm)}\\
\textbf{Hướng dẫn sử dụng} \textit{(0,5 điểm)}\\
\textbf{Kết quả thực hiện đúng với kết quả của phần ứng dụng} \textit{(tối đa 3,5 điểm)}
\begin{itemize}[itemsep=-6pt, topsep=-6pt]
  \item Kết quả đúng \textit{(2,0 điểm)}
  \item Cách thức thực hiện hợp lý \textit{(1,0 điểm)}
  \item Chức năng bổ sung, sáng tạo \textit{(0,5 điểm)}
\end{itemize}

\begin{flushright}
  \begin{tabular}{@{}c@{}}
    Cần Thơ, ngày ........ tháng ........ năm 2025 \\
    GIÁO VIÊN CHẤM                                 \\
    \\
    \\
    \\
    Trương Thị Thanh Tuyền                         \\
  \end{tabular}
\end{flushright}
\normalsize