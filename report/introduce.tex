\phantomsection
\setsection{Chương 1: Giới thiệu}
\setcounter{section}{1}

% \section*{\centering CHƯƠNG 1: TỔNG QUAN}
\phantomsection
\subsection{Mục tiêu}
Mục đích của tài liệu Đặc tả Yêu cầu Phần mềm (SRS) này là xác định các yêu cầu phần mềm cho **Hệ thống Quản lý Nhà hàng CUKCUK**, phiên bản 1.0. Tài liệu này mô tả các tính năng và chức năng cần thiết để quản lý các khía cạnh khác nhau của hoạt động nhà hàng, bao gồm quản lý đặt chỗ, xử lý đơn hàng, phối hợp trong bếp, xử lý thanh toán và quản trị kinh doanh tổng thể. Tài liệu này tập trung vào việc mô tả các yêu cầu phần mềm cho cả phần mềm mà nhân viên nhà hàng sẽ sử dụng (như giao diện dành cho nhân viên phục vụ và bếp) và phần mềm quản lý dành cho người quản lý và chủ nhà hàng để họ có thể kiểm soát và vận hành nhà hàng một cách hiệu quả

Phạm vi của tài liệu này bao gồm tất cả các thành phần cốt lõi của hệ thống cần thiết để hỗ trợ quản lý nhà hàng hiệu quả nhưng không bao gồm tích hợp với các hệ thống bên thứ ba và các phân hệ ngoại vi như phần mềm kế toán bên ngoài.
\phantomsection

\subsection{Quy ước của tài liệu}
Kiểu chữ: Time New Roman

Kích cỡ chữ: 13

Chiều dọc theo giấy A4
\phantomsection

\phantomsection

\subsection{Nhóm những người sử dụng}
Tài liệu Đặc tả Yêu cầu Phần mềm (SRS) này được dành cho nhiều loại độc giả khác nhau có liên quan đến việc phát triển và vận hành Hệ thống Quản lý Nhà hàng CUKCUK. Đối tượng chính bao gồm nhà phát triển, những người sẽ sử dụng tài liệu này để hiểu rõ các yêu cầu kỹ thuật; quản lý dự án, những người sẽ tham khảo tài liệu để lập kế hoạch và theo dõi tiến độ dự án; nhân viên marketing, những người sẽ nắm bắt được các tính năng của hệ thống; người dùng, đặc biệt là quản trị viên và nhân viên nhà hàng, những người sẽ thấy lợi ích từ việc hiểu các chức năng của hệ thống; người kiểm thử, những người sẽ sử dụng tài liệu để xác minh rằng tất cả các yêu cầu đều được đáp ứng; và người viết tài liệu, chịu trách nhiệm tạo ra hướng dẫn sử dụng và các tài liệu hướng dẫn dựa trên các tính năng đã được xác định.
\phantomsection

\subsection{Phạm vi sản phẩm}
Phần mềm được mô tả trong tài liệu này là Hệ thống Quản lý Nhà hàng CUKCUK, với mục đích giúp các nhà hàng tối ưu hóa quy trình quản lý và vận hành, từ việc đặt chỗ, ghi order, quản lý chế biến, đến xử lý thanh toán và quản trị tài chính. Phần mềm mang lại những lợi ích quan trọng như giảm thiểu sai sót, tiết kiệm thời gian phục vụ, cải thiện trải nghiệm khách hàng, và tăng hiệu suất làm việc của nhân viên. Mục tiêu chính của hệ thống là hỗ trợ các nhà hàng vận hành hiệu quả hơn, từ quy mô nhỏ lẻ đến chuỗi nhà hàng lớn. Đồng thời, phần mềm này còn hỗ trợ nhà quản lý theo dõi, kiểm soát toàn diện hoạt động kinh doanh theo thời gian thực, từ xa hoặc trực tiếp. Phần mềm liên kết chặt chẽ với các mục tiêu kinh doanh của doanh nghiệp, giúp tăng doanh thu, cải thiện sự hài lòng của khách hàng, và hỗ trợ việc ra quyết định dựa trên dữ liệu cụ thể.
\phantomsection

\subsection{Phạm vi sản phẩm}
Phần mềm được mô tả trong tài liệu này là Hệ thống Quản lý Nhà hàng CUKCUK, với mục đích giúp các nhà hàng tối ưu hóa quy trình quản lý và vận hành, từ việc đặt chỗ, ghi order, quản lý chế biến, đến xử lý thanh toán và quản trị tài chính. Phần mềm mang lại những lợi ích quan trọng như giảm thiểu sai sót, tiết kiệm thời gian phục vụ, cải thiện trải nghiệm khách hàng, và tăng hiệu suất làm việc của nhân viên. Mục tiêu chính của hệ thống là hỗ trợ các nhà hàng vận hành hiệu quả hơn, từ quy mô nhỏ lẻ đến chuỗi nhà hàng lớn. Đồng thời, phần mềm này còn hỗ trợ nhà quản lý theo dõi, kiểm soát toàn diện hoạt động kinh doanh theo thời gian thực, từ xa hoặc trực tiếp. Phần mềm liên kết chặt chẽ với các mục tiêu kinh doanh của doanh nghiệp, giúp tăng doanh thu, cải thiện sự hài lòng của khách hàng, và hỗ trợ việc ra quyết định dựa trên dữ liệu cụ thể.
\phantomsection
